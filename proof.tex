%% LyX 2.0.5 created this file.  For more info, see http://www.lyx.org/.
%% Do not edit unless you really know what you are doing.
\documentclass{article}
\usepackage[T1]{fontenc}
\usepackage[latin9]{inputenc}
\usepackage{geometry}
\geometry{verbose,lmargin=2cm,rmargin=2cm}
\begin{document}
Our goal here is to show that the subset of problems solved in polynomial
time by WL is strictly included in the subset solved by our heuristic.
We shall consider an extended version of WL (EWL) to facilitate the
proof, where colors aren't redistributed among $[1;n]$ but are instead
an injective function of the vertex's previous color and the multiset
of its neighbour's colors into the set of colors (which is not anymore
included in $[1;n]$). Two cases allow WL to reduce the number of
possibilities and hence have a polynomial runtime : the first is when
the colorings of G and G' are incompatible, and the second when there
is a color shared by an unique vertex in each graph. 

We must show three properties : 
\begin{itemize}
\item When the graphs are isomorphic, WL, EWL and PN compute the isomorphism,
and produce a negative certificate in the other case (correctness)
\item When WL finds two incompatible coloring, then so does EWL, and PN
also gives a negative certificate (polynomial negative certificate)
\item When WL colors a vertex in an unique color in both graphs, then EWL
does the same, and our algorithm also recognizes that those two vertices
must be linked if the graphs are isomorphic (polynomial isomorphism)
\end{itemize}
The first is immediate, because the operations we use are invariant
through permutations of the vertices, hence if the two graphs are
isomorphic it might take exponential time but will surely end. The
two other properties trivially hold for extended WL, and we must show
that it implies that they hold for PN. However, there is an advantage
of EWL over WL, because the coloring is unique for the two graphs
(it being injective), so if there is exactly one vertex in each graph
colored by C, then the isomorphism must assign one to the other (problems
may arise in WL because the coloring is not necessarily injective).
This gives an easy proof of the third depending on the second, because
the fact that a vertex is colored uniquely in G and G' is implied
by the fact that it is colored differently from every single other
vertex. Hence we must only prove the second property.

We shall proceed by induction to prove that if EWL colors a vertice
in G and another in G' in different colors, then PN also allows us
to differentiate those two vertices. At the first run of the coloring,
every vertex is colored by its degree so it is trivially true. 

Now let us suppose that the property holds at run n, and show that
it holds at run n+1 : 

Consider two vertices V and W (one per graph) that had the same color
for all previous runs, but which are colored differently at run n+1.
Then their neighbourhoods are incompatible, which means that the multiset
of colors is not the same, so there is a color c such that there is
one more vertex of color c in N(V) than in N(W). However, those colors
were attributed at run n, so it means that when comparing the PN lists
of neighbours of V and W, those lists will be incompatible by hypothesis.
Hence V and W can't be linked in PN, which ends the induction. 
\end{document}
