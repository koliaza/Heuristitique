%% LyX 2.0.5 created this file.  For more info, see http://www.lyx.org/.
%% Do not edit unless you really know what you are doing.
\documentclass{article}
\usepackage{mathpazo}
\usepackage[T1]{fontenc}
\usepackage[latin9]{inputenc}
\usepackage{geometry}
\geometry{verbose,lmargin=2cm,rmargin=2cm}
\begin{document}

\title{An heuristic for graph isomorphisms}


\author{Nguy\~{\^e}n L� Th�nh D\~ung \and  Blanchard Nicolas Koliaza}

\maketitle

\section*{Foreword }

The problem considered is the graph isomorphism (GI) problem : given
two graphs $G$ and $G'$, can one compute an isomorphism between
them. We implement our own heuristic called PN for Path-Neighbour.
This is the short version of the report, and more details can be found
in the complete version.


\section{The Heuristics}


\subsection{The Weisfeiler-Lehman heuristic}

The original Weisfeiler-Lehman (WL) heuristic works by coloring the
edges of a graph according to the following rules :
\begin{itemize}
\item We begin with a coloring that assigns to every vertex the same color
(this is the 1-dimensional version).
\item At each pass, the color of each vertex is determined by the number
of neighbours of color c for each c.
\item After at most n passes, the colors don't change anymore. We then make
one random choice and use backtracking before coloring again. 
\end{itemize}
Two isomorphic graphs admit the same coloring, but the converse does
not always hold: k-regular graphs for example are pathological cases
which take exponential time.


\subsection{The PN heuristic}


\subsubsection{The idea}

PN is based on the following property : 

Let $N_{k}(x)$ be the number of neighbours at distance exactly k
from x, and $P_{k}(x)$ the number of paths of length k starting from
x, then if $f$ is an isomorphism between $G$ and $G'$ , $N_{k}(x)=N_{k}(f(x))$
and $P_{k}(x)=P_{k}(f(x))$. Thus, by computing the different $N_{x}$
and $P_{x}$ we can prune the search tree and limit the possibilities.
We name the array of couples $P_{k}(x),N_{k}(x)$ for k between 1
and n PN(x), and compute an array containing PN(x) for each x, obtaining
the PN-arrays. 


\subsubsection{Structure of the algorithm}

The algorithm we use actually incorporates multiple testing phases
to quickly eliminate easy cases. It can be decomposed in the following
steps : 
\begin{enumerate}
\item Input parsing and choice of data structure
\item Primary test phase (a collection of fast tests to quickly eliminate
easy cases)
\item Construction and sorting of each PN-array 
\item Comparison of the PN-arrays using the neighbours
\item If possible construction of an isomorphism by refined bruteforce
\end{enumerate}

\section{Comparison with Weisfeiler-Lehman}


\subsection{Proof}

We can show that PN behaves polynomially on a subset of problems that
strictly includes the subset where WL behaves polynomially. We consider
an extended version of WL which is very similar and facilitates the
analysis (even though it is much less efficient in practice). The
proof is done by considering the cases when WL prunes efficiently
the search tree and by showing by induction that PN does the same
in those cases. The detailed proof is in the long report. 


\subsection{Complexity analysis}

As the GI problem isn't known to be in P, it is not absurd to have
a worst running time of $O(n!)$. However the running time is generally
lower, and PN often takes $O(n^{4})$. With the primary test phase,
it can go down to $O(n^{2}m)$ or even $O(m+n*log(n))$ in some cases.
This means that WL (which can run in $O(nm)$) is often more efficient,
but PN is safer (the pathological cases are not as frequent).


\subsection{Problems, optimization, and improvements}

The biggest problem was that the numbers in the PN-arrays quickly
grow out of proportions so one has to to multiplication modulo p,
which means that our implementation is only probabilistic. PN might
be improved by using adjacency lists in the case of sparse graphs,
but this would require a lot of new code for an improvement that might
not be sizeable. Some other improvements would target optimization
of the mulp function, parallelization inside the matrix multiplication
and interlacing neighbourhood checks with PN-array generation.
\end{document}
