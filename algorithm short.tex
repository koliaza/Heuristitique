%% LyX 2.0.5 created this file.  For more info, see http://www.lyx.org/.
%% Do not edit unless you really know what you are doing.
\documentclass{article}
\usepackage{mathpazo}
\usepackage[T1]{fontenc}
\usepackage[latin9]{inputenc}
\usepackage{geometry}
\geometry{verbose,lmargin=2cm,rmargin=2cm}
\begin{document}

\title{An heuristic for graph isomorphisms}


\author{Nguy\~{\^e}n L� Th�nh D\~ung \and  Blanchard Nicolas Koliaza}

\maketitle

\section*{Foreword }

The problem considered is the graph isomorphism (GI) problem : given
two graphs $G$ and $G'$, can one compute an isomorphism between
them. We implement our own heuristic called PN for Path-Neighbour.
This is the short version of the report, and more details can be found
in the complete version.


\section{The Heuristics}


\subsection{The Weisfeiler-Lehman heuristic}

The original Weisfeiler-Lehman (WL) heuristic works by coloring the
edges of a graph according to the following rules :
\begin{itemize}
\item We begin with a coloring that assigns to every vertex the same color
(this is the 1-dimensional version).
\item At each pass, the color of each vertex is determined by the number
of neighbours of color c for each c.
\item After at most n passes, the colors don't change anymore.
\end{itemize}
Two isomorphic graphs admit the same coloring, but the converse does
not always hold: k-regular graphs for example are pathological cases
which take exponential time.


\subsection{The PN heuristic}


\subsubsection{The idea}

PN is based on the following property : 

Let $N_{k}(x)$ be the number of neighbours at distance exactly k
from x, and $P_{k}(x)$ the number of paths of length k starting from
x, then if $f$ is an isomorphism between $G$ and $G'$ , $N_{k}(x)=N_{k}(f(x))$
and $P_{k}(x)=P_{k}(f(x))$. Thus, by computing the different $N_{x}$
and $P_{x}$ we can prune the search tree and limit the possibilities.
We name the array of couples $P_{k}(x),N_{k}(x)$ for k between 1
and n PN(x), and compute an array containing PN(x) for each x, obtaining
the PN arrays. 


\subsubsection{Structure of the algorithm}

The algorithm we use actually incorporates multiple testing phases
to quickly eliminate easy cases. It can be decomposed in the following
steps : 
\begin{enumerate}
\item Input parsing and choice of data structure
\item Primary test phase (a collection of fast tests to quickly eliminate
easy cases)
\item Construction and sorting of each PN-array 
\item Comparison of the PN-arrays using the neighbours
\item If possible construction of an isomorphism by refined bruteforce
\end{enumerate}

\subsection{Tests and Benchmark}

We have four test generation programs, to create G(n,p) graphs, G(n,m)
graphs, k-regular graphs and to create an isomorphic copy of any graph. 


\section{Comparison with Weisfeiler-Lehman}


\subsection{Proof}

We can show that PN behaves polynomially on a subset of problems that
strictly includes the subset where WL behaves polynomially. We consider
an extended version of WL which is very similar and facilitates the
analysis (even though it is much less efficient in practice). The
proof is done by considering the cases when WL prunes efficiently
the search tree and by showing by induction that PN does the same
in those cases. 


\subsection{Complexity analysis}

As the GI problem isn't known to be in P, it is not absurd to have
a worst running time of $O(n!)$. However the running time is generally
lower, and PN often takes $O(n^{4})$. With the primary test phase,
it can go down to $O(n^{2}m)$ or even $O(m+n*log(n))$ in some cases.
This means that WL (which can run in $O(nm)$) is often more efficient,
but PN is safer (the pathological cases are not as frequent).


\section{Implementation and Optimization}


\subsection{Data Structures}

The choice of data structures varies between adjacency matrix and
list, and sometimes both, and allows us to take advantage of faster
algorithms on sparse graphs using lists, while allowing us to know
in O(1) if two vertices are linked. 

The construction of the PN-array is generally the most time-expensive
task, it is mostly done by multiplication of the adjacency matrix.
Sorting the PN-array is easily done with a heapsort. 

The construction of the isomorphism is done by trying to assign to
an x an y such that$N(x)=N(y)$ and backtracking when unsuccessful. 


\subsection{Function Optimization}

As with many graph algorithms, PN can be improved by using adjacency
lists in the case of sparse graphs. However, that seldom changes the
asymptotic complexity, as the PN-array generation can only benefit
from it in case of graphs with very poor expansion properties. The
list representation is mostly used in the primary test phase, where
it allows us to have tests in $O(m)$ instead of $O(n^{2})$. Optimization
on the matrix operations cannot really be done in practice as Strassen's
algorithm has no interest on the matrix size we generally consider.
However we took care of memory locality in their representation, which
might improve the performance in practice. 


\subsection{Parallelism}

The advantage of the PN-arrays is that they can be generated in a
completely independent fashion, and so are easy to parallelize. By
default we create a different thread for each graph, which gives a
speed-up ratio close to 2 in the basic case, but that can grow up
to the number of cores when launched on a set of graphs. This is an
advantage of PN over WL as the threads do not have to communicate,
while the latter has to do some comparisons between each coloring
at each run, requiring constant interaction between threads.
\end{document}
